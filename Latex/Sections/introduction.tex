The dataset we've chosen to use represents a collection of reviews of products sold on Amazon. 
Datasets like this are usually used for \textbf{analytic purposes}, such as understanding the customers' needs and preferences, for \textbf{marketing purposes}, such as collecting the customers' opinions on a product, or for \textbf{sentiment analysis}, such as training a classifier to predict the sentiment of a review basing on its content, since information about the review's text is also provided.

In our case, we will consider the first task, since it can be reduced to the problem of writing a suited set of queries and can be also efficiently supported by the database technology we decided to use, which is MongoDB.
We've decided to interpret the task of strategic analysis as the problem of extracting meaningful statistics from the dataset (such as the number of reviews per product, the average rating per product, the review with highest score, ...) which could be then included on the company's reports and used to make strategic decisions.\\

One of the main reasons behind our technical choice of using MongoDB as database technology is the fact that the dataset is in JSON format, which is natively supported by MongoDB. 
Moreover, it is composed of a collection of, possibly, nested documents, which is also a good fit for MongoDB, since it is a document-oriented database, meaning that it is specifically designed to store data as documents instead of relational tables.